\documentclass{beamer}
\usepackage{listings}
\usepackage{color}
\usepackage[T1]{fontenc}


\usetheme{Frankfurt}
\usecolortheme{crane}

\title{Esame Telerilevamento Geoecologico}
\institute{Alma Mater Studiorum - Università di Bologna\\Telerilevamento Geo-Ecologico}
\author{Studente: Matteo Barbero\\Docente: Duccio Rocchini}
\date{A.A. 2021/2022}
\logo{\includegraphics[height=1cm]{logo.png}}


\begin{document}

\maketitle

\AtBeginSection[]
{
\begin{frame}
\frametitle{Indice}
\tableofcontents[currentsection, currentsubsection]
\end{frame}
}
\AtBeginSubsection[] {
    \begin{frame}
    \frametitle{Indice} 
    \tableofcontents[currentsubsection]  
    \end{frame}
}


\section{Scopo del progetto}

\begin{frame}{Scopo del progetto}
\begin{itemize}
    \item  Mostrare il cambiamento dell'uso del suolo in una porzione di territorio italiano, in un intervallo di tempo di 38 anni (1984-2022).
\end{itemize}
\end{frame}


\section{Inquadramento geografico}

\begin{frame}{Inquadramento geografico}
    \includegraphics[width=1\textwidth]{0-area.jpg}
    \centering
\end{frame}


\section{Metodi di analisi}

\begin{frame}{Metodi di analisi}
\begin{itemize}
    \item Preparazione iniziale delle immagini;
    
    \bigskip
    
    \item \pause Calcolo degl'indici spettrali delle due immagini;
    
    \bigskip
    
    \item \pause Classificazione delle due immagini.
\end{itemize}    
\end{frame}


\subsection{Preparazione iniziale delle immagini}

\begin{frame}{Script usato}
    \begin{tiny}
        \lstinputlisting[language=R]{Script5.R}
    \end{tiny}
\end{frame}

\begin{frame}{1984}
    \includegraphics[width=1\textwidth]{1-H1984.jpg}
    \centering
\end{frame}

\begin{frame}{2022}
    \includegraphics[width=1\textwidth]{2-H2022.jpg}
    \centering
\end{frame}


\subsection{Indici spettrali}

\begin{frame}{Indici spettrali}
    \begin{itemize}
        \item Gli indici che sono stati utilizzati sono il \textbf{DVI} e l'\textbf{NDVI};
        
        \bigskip
        
        \item \pause DVI:  difference vegetation index;
        
        \bigskip
        
        \item \pause NDVI:  normalized difference vegetation index.
    \end{itemize}
\end{frame}

\begin{frame}{Script usato}
    \begin{scriptsize}
    \lstinputlisting[language=R]{Script1.R}
    \end{scriptsize}
\end{frame}

\begin{frame}{DVI}
    \begin{columns}
    \begin{column}{0.6\textwidth}
    \begin{figure}
        \centering
        \includegraphics[width=0.9\textwidth]{pardvi.jpg}
    \end{figure}
    \end{column}
    \begin{column}{0.1\textwidth}
     1984 
     
     \bigskip
     
     \bigskip
     
     \bigskip
     
     \bigskip
     
     \bigskip
     
     \bigskip
     
     \bigskip
     
     \bigskip
     
     \bigskip
     
     \bigskip
     
     2022
    \end{column}
    \end{columns}
\end{frame}

\begin{frame}{DIFDVI}
    \includegraphics[width=1\textwidth]{5-difdvi.jpg}
    \centering
\end{frame}

\begin{frame}{Script usato}
    \begin{scriptsize}
        \lstinputlisting[language=R]{Script2.R}
    \end{scriptsize}
\end{frame}


\begin{frame}{NDVI}
    \begin{columns}
    \begin{column}{0.6\textwidth}
    \begin{figure}
        \centering
        \includegraphics[width=0.9\textwidth]{parndvi.jpg}
    \end{figure}
    \end{column}
    \begin{column}{0.1\textwidth}

    1984 
     
     \bigskip
     
     \bigskip
     
     \bigskip
     
     \bigskip
     
     \bigskip
     
     \bigskip
     
     \bigskip
     
     \bigskip
     
     \bigskip
     
     \bigskip
     
    2022

    \end{column}
    \end{columns}
\end{frame}

\begin{frame}{DIFNDVI}
    \includegraphics[width=1\textwidth]{8-difndvi.jpg}
    \centering
\end{frame}


\subsection{Classificazione dell'immagine}

\begin{frame}{Classificazione dell'immagine}
\begin{itemize}
    \item L'immagine viene classificata, in n classi, in base alla riflettanza di ogni pixel dell'immagine sia nella banda del \textbf{"rosso (RED)"} che nella banda del \textbf{"vicino infrarosso (NIR)"};
    
    \bigskip
    
    \item \pause La classificazione avviene tramite un algoritmo (\textbf{Maximum Likelihood});
    
    \bigskip
    
    \item \pause La funzione usata per classificare le immagini si chiama:  \textbf{unsuperClass} (classificazione non supervisionata).
\end{itemize}
\end{frame}

\begin{frame}{Script usato}
    \begin{scriptsize}
        \lstinputlisting[language=R]{Script3.R}
    \end{scriptsize}
\end{frame}

\begin{frame}{Immagine 1984 classificata}
    \includegraphics[width=1\textwidth]{9-class1984.jpg}
    \centering
    
    \medskip
    
1-Altro 2-Foresta 3-Agricoltura 4-Acqua
\end{frame}

\begin{frame}{Immagine 2022 classificata}
    \includegraphics[width=1\textwidth]{10-class2022.jpg}
    \centering
    
    \medskip
    
    1-Acqua 2-Altro 3-Foresta 4-Agricoltura
\end{frame}

\begin{frame}{Script usato}
    \begin{tiny}
        \lstinputlisting[language=R]{Script4.R}
    \end{tiny}
\end{frame}

\begin{frame}{Tabella con le percentuali}
    \includegraphics[width=1\textwidth]{13-tabella.jpg}
    \centering
\end{frame}

\begin{frame}{Istogramma delle classificazioni}
    \includegraphics[width=1\textwidth]{11-histo.jpg}
    \centering
\end{frame}

\section{Conclusioni}

\begin{frame}{Conclusioni}
    Negli ultimi 40 anni, in questa zona, c'è stata una tendenza alla bonifica e al disboscamento (seppur in maniera ridotta) in favore dell'agricoltura e dell'urbanizzazione. 
\end{frame}

\begin{frame}
\begin{Huge}
GRAZIE PER L'ATTENZIONE
\end{Huge}
\end{frame}

\end{document}
