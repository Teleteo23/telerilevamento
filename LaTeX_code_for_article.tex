% i commenti si fanno con la percentuale
\documentclass[a4paper,12pt]{article} %classe utilizzata per specificare il tipo di documento
\usepackage[utf8]{inputenc}
\usepackage{graphicx} % pacchetto per parti grafiche
% con le parentesi quadre si mettono gli imput
\usepackage{color}
\usepackage{lineno}
\usepackage{hyperref}
\usepackage{natbib}
\usepackage{listings}
\usepackage{setspace}

\newcommand{\tr}{\textcolor{red}}

\title{Il mio primo documento in LaTeX}
\author{Matteo Barbero}
\date{2022/2023} %commentando la data viene messa in automatico prendendola dal pc

\begin{document} % alla fine di tutto si deve mettere un "end"

\maketitle
\doublespacing
\tableofcontents

\begin{abstract}
Here is the abstract of my Master Thesis.

The Master Thesis is dealing with something.

Concluding, I made beautiful science.
\end{abstract}

\noindent \textbf{Keywords}: biodiversity; Brenta chain; geology; remote sensing; spatial ecology; statistics

\section{Introduction}\label{sec:intro}

Since the 1920s, \textcolor{red}{aerial photography} has represented an important data source for the detection of landscape patterns and their change over time. \\

% \smallskip
% \bigskip

Multitemporal analysis represents a powerful method for the study of all ecological and geological processes that change over time. The literature involves several fields of study: from soil loss to natural resources assessment to vegetation and ecological dynamics.

\section{Area di studio}

The study area is the nature reserve of Poggio all’Olmo in Tuscany (Figure \ref{fig:topsoil}), Italy (11 28 26E, 42 51 45N, WGS84 Datum). It is located on the side of Mt. Amiata,
comprising 440 ha, with elevations ranging from 650 to 1016 m above mean sea level (m.s.l.) and slopes from 0 to 55

\begin{figure}
\centering
\includegraphics[width=0.5\textwidth]{Confronto topsoil.png}
\caption{grafico.}
\label{fig:topsoil}
\end{figure}

\begin{figure}[h]
        \centering
        \includegraphics[width=0.5\textwidth]{Confronto topsoil.png}
        \caption{Caption}
        \label{fig:my_label}
\end{figure}

\section{Metodi}

Sono divise in:
\begin{itemize}
    \item The formulas
    \item The code
\end{itemize}

\begin{enumerate}
    \item formule
    \item codice
\end{enumerate}

\subsection{formule}
equation \ref{eq:newton}
\begin{equation}
   F = G \times \frac{m_{1} \times m_{2}}{d^{2} \times}
   \label{eq:newton}
\end{equation}

\begin{equation}
    F = \sqrt{G \times \frac{m_{1} \times m_{2}}{d^{2}}}
\end{equation}

\begin{equation}
     F = \frac{\sqrt[3]{G \times \frac{m_{1} \times m_{2}}{d^{2}}}}{-\sum{p(x) \times \log{p(x)}}}
\end{equation}
\subsection{codice}

\lstinputlisting[language=R]{Script1.r}

\section{Risultati}

As a result of this study I found that the Cadmium is present in the analyzed soil with a total amount of 15\%.
Let's put a formula directly in the main text. We can apply this: $F=G \times m_{1}$. These results were achieved according to Equation \ref{eq:newton}.

\section{Discussioni}

Come detto nella sezione \ref{sec:intro}

Come detto da \citep{Revell_2012, Potts_2022}

Spiegato in \citep{Massatti_2022}

\begin{thebibliography}{999}
\bibitem[Revell(2012)]{Revell_2012}
Revell, L.J. (2012), phytools: an R package for phylogenetic comparative biology (and other things). Methods in Ecology and Evolution, 3: 217-223. \url{https://doi.org/10.1111/j.2041-210X.2011.00169.x}

\bibitem[Massatti and Winkler(2022)]{Massatti_2022}
Massatti, R. and Winkler, D.E. (2022), Spatially explicit management of genetic diversity using ancestry probability surfaces. Methods Ecol Evol. Accepted Author Manuscript. \url{https://doi.org/10.1111/2041-210X.13902}

\bibitem[Potts et al.(2022)]{Potts_2022}
Potts, J.R., Börger, L., Strickland, B.K. and Street, G.M. (2022), Assessing the predictive power of step selection functions: how social and environmental interactions affect animal space use. Methods Ecol Evol. Accepted Author Manuscript. \url{https://doi.org/10.1111/2041-210X.13904}

\end{thebibliography}

\end{document}
